\documentclass[11pt]{article}

% -------- Español ---------
\usepackage[spanish]{babel}
\selectlanguage{spanish}
\usepackage[utf8]{inputenc}
% --------------------------

% ------- No numerar -------
\makeatletter
\def\@seccntformat#1{%
  \expandafter\ifx\csname c@#1\endcsname\c@section\else
  \csname the#1\endcsname\quad
  \fi}
\makeatother
% --------------------------

% ------ Listas formato ----
\usepackage{enumitem}
% --------------------------

\begin{document}
\title{Propuesta de tesis: Inferencia bayesiana a gran escala con \textit{Stein
Variational Gradient Descent}}
\author{Jorge Rotter}
\maketitle

\section{Resumen}

Recientemente, mucha de la investigación –- y de las aplicaciones --  del
aprendizaje de máquina se ha enfocado en el \textit{deep learning}, (redes
neuronales profundas que se han probado exitosas en reconocimiento de imágenes
y análisis de texto) y en algoritmos para grandes volúmenes de datos. Sin
embargo, casi todos los algoritmos de aprendizaje modernos utilizan técnicas de
optimización  que dificultan cuantificar la incertidumbre. Más aún, incluso los
modelos bayesianos más clásicos son difíciles de escalar a la complejidad (en
parámetros y formas funcionales) y el volumen de los datos que han dado éxito a
su contraparte frecuentista. En este trabajo se explorará un algoritmo de
inferencia variacional propuesto por Liu y Wang \cite{svgd} \textit{Stein
Variational Gradient Descent} (SVGD). Este algoritmo mezcla la flexibilidad
no-paramétrica de los métodos MCMC con la facilidad de cómputo de los métodos
variacionales. \\

SVGD plantea aproximar la posterior con partículas que se actualizan
iterativamente para reducir la divergencia de Kullback-Leibler desde su medida
empírica. Para facilitar el cómputo, elige como espacio para la optimización la
bola unitaria de un \textit{reproducing kernel hilbert space} (RKHS) y utiliza
un resultado que liga la derivada de la divergencia de Kullback-Leibler con la
discrepancia de Stein, que surge en el método de Stein utilizado para acotar
aproximaciones de mometos. Mostraremos que el algoritmo puede interpretarse
como una forma de descenso en gradiente funcional en este RKHS, pero también
como un flujo de gradiente en un espacio de distribuciones (cuando las
partículas se van a infinito) y como un método para que la esperanza de todas
las funciones en cierto conjunto relativo a la medida empírica de las
partículas coincida con la de relativa a la posterior. Esta última
interpretación además pone en perspectiva la elección del kernel y el efecto
que tiene sobre la convergencia del algoritmo.\\

En las últimas secciones se implementará SVGD para probarlo contra MCMC en un
modelo de mezclas y contra contra \textit{backpropagation} en una red neuronal,
la forma usual de entrenarlas.
\section{Propuesta de contenidos}
\begin{enumerate}
	\item Motivación
	\begin{enumerate}[label=1.\arabic*]
\item El estado actual del aprendizaje de máquina:  grandes volúmenes,
\textit{deep learning} y \textit{probabilistic programming}
		\item Inferencia bayesiana a gran escala
	\end{enumerate}
	\item Inferencia variacional
	\begin{enumerate}[label=2.\arabic*]
		\item La divergencia de Kullback-Leibler
		\item Inferencia variacional \cite{vi}
\item Inferencia variacional con diferenciación automática (ADVI)
\cite{Kucukelbir2016}
	\end{enumerate}
	\item Reproducing Kernel Hilbert Spaces \cite{Berlinet2009}
	\begin{enumerate}[label=3.\arabic*]
\item Equivalencia entre definiciones %(incluyendo teorema de Riesz y de
Moore-Aronszajn)
		\item Kérneles vía \textit{random features} \cite{Rahimi}
	\end{enumerate}
	\item Stein Variational Gradient descent
	\begin{enumerate}[label=4.\arabic*]
\item La discrepancia de Stein kernelizada \cite{kernelized-stein-discrepancy,
measuring-quality}
		\item Construcción del algoritmo SVGD \cite{svgd}
	\end{enumerate}
	\item Análisis de convergencia 
	\begin{enumerate}[label=5.\arabic*]
		\item Análisis asintótico \cite{svgd-gradient-flow}
		\begin{enumerate}[label=5.1.\arabic*]
			\item Medidas empíricas
			\item Convergencia (débil) en el número de partículas
			\item Convergencia (débil) en el tiempo
			\item Interpretación geométrica y transporte óptimo 
		\end{enumerate}
		\item Análisis en partículas finitas \cite{svgd-moment-matching}
		\begin{enumerate}[label=5.2.\arabic*]
			\item Punto fijo de SVGD
			\item SVGD con \textit{feature maps}
			\item SVGD con \textit{features} aleatorias
		\end{enumerate}

	\end{enumerate}
	\item Implementación y pruebas
	\item Conclusiones
\end{enumerate}

\bibliographystyle{unsrt}
\bibliography{referencias}


\end{document}
