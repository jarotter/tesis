\documentclass[main.tex]{subfiles}
\begin{document}

\chapter{Medida, integración y probabilidad}

En este capítulo revisamos algunos conceptos de teoría de la medida y su aplicación a la probabilidad. Las definiciones dan por hecho existencia y unicidad y se omiten las pruebas de todos los resultados. Referimos a \cite{halmos1949}, \cite{durrett} y \cite{ash}.
\begin{definition}
Sea $\Omega\neq\emptyset$. Una \textit{$\sigma$-álgebra en $\Omega$} es una colección $\mathcal{S}\subseteq\mathcal{P}(\Omega)$ tal que
\begin{enumerate}[label=\roman*.]
	\item $X \in \mathcal{S}$
	\item Si $A\in\mathcal{S}$, entonces $\Omega\setminus A \in \mathcal{S}$.
	\item Si $(A_n)_{n=1}^\infty$ es una sucesión en $\mathcal{S}$, entonces $\cup_nA_n\in \mathcal{S}$.
\end{enumerate}
La pareja $(\Omega, \mathcal{S})$ es un \textit{espacio medible}. A los elementos $A\in\mathcal{S}$ se les dice \textit{conjuntos medibles} de este espacio.
\end{definition}

De las leyes de De Morgan se sigue que una $\sigma$-álgebra también es cerrada bajo intersecciones contables. 

En analogía con la continuidad topológica, las funciones que preservan estructura reciben un nombre especial.

\begin{definition}
Sean $(\Omega, \mathcal{S})$ y $(\Xi, \mathcal{E})$ espacios medibles. Una función $f: \Omega \to \Xi$ es $\mathcal{F,E}$\textit{-medible} si para todo $B\in\mathcal{E}$
\begin{equation*}
\pim_f(B)\in\mathcal{S}	
\end{equation*}
Es decir, si la preimagen de cada medible es medible. Cuando no se preste a ambigüedad, omitimos las $\sigma$-álgebras para simplificar la notación.
\end{definition}

\begin{definition}
Sea $(\Omega, \tau)$ un espacio topológico. La $\sigma$\textit{-álgebra de Borel} de este espacio es la mínima $\sigma$-álgebra en $\Omega$ que contiene a $\tau$. Se le denota $\mathbb{B}(\Omega, \tau)$.
En el caso particular de $\mathbb{R}^d$ con su topología usual, usamos simplemente $\mathbb{B}^d$.
\end{definition}

La idea detrás de esta definición es más general, y puede hablarse de la mínima $\sigma$-álgebra que contenga a cualquier colección de conjuntos $\mathcal{E}$. En este caso, llamamos a $\mathcal{E}$ \textit{conjunto generador}, y denotamos a la sigma álgebra $\sigma(\mathcal{E})$.

Los espacios medibles son importantes porque en ellos puede definirse una buena forma de cuantificar volumen.

\begin{definition}
Sea $(\Omega, \mathcal{S})$ un espacio medible. Una \textit{medida} en $\Omega$ es una función $\mu: \mathcal{S} \to \overline{\mathbb{R}}$ tal que
\begin{enumerate}[label=\roman*.]
	\item $\mu(A)\geq \mu(\emptyset)=0$ para todo $A\in\mathcal{S}$
	\item Si $(A_n)_{n=1}^\infty$ es una sucesión de conjuntos disjuntos en $\mathcal{S}$, 
	\begin{equation*}
	\mu\left(\bigcup_{\ n\in\mathbb{N}}A_n\right)=
	\sum_{n\in\mathbb{N}}\mu(A_n)	
	\end{equation*}
\end{enumerate}
La terna $(\Omega, \mathcal{S}, \mu)$ recibe el nombre de \textit{espacio de medida}. Si $\mu(\Omega)=1$, la función $\mu$ es una \textit{medida de probabilidad} y $(\Omega, \mathcal{S}, \mu)$ un \textit{espacio de probabilidad}. De forma genérica, $P$ y $Q$ denotarán medidas de probabilidad.
\end{definition}

\begin{definition}
Una propiedad se satisface \textit{casi dondequiera relativo a} $\mu$, denotado $\mu$-\textit{c.d} si se satisface para toda $\omega\in \Omega\setminus A$ y $\mu(A)=0$. Cuando no se preste a ambigüedad  omitimos la medida.
\end{definition}

Un ejemplo importante de medida es el volumen como se entiende en la vida cotidiana en $\mathbb{R}^3$, área en $\mathbb{R}^2$ y longitud en $\mathbb{R}$. De hecho fue para satisfacer estas propiedades es que necesitó limitarse el dominio de una medida a un sigma álgebra y no todo $\mathcal{P}(\Omega)$. 

\begin{definition}
	La \textit{medida de Lebesgue en $\mathbb{R}^d$} es la única medida \\ $\lambda:\mathbb{B}^d\to\overline{\mathbb{R}}$ tal que
	\begin{equation*}
		\lambda\left(\bigtimes_{i=1}^d (a_i, b_i]\right)	= \prod_{i=1}^d(b_i-a_i)
	\end{equation*}
	Equivalentemente, es la única medida en este espacio invariante ante traslaciones y rotaciones. 
\end{definition}

En un curso de cálculo integral se introduce la integral de Riemann como herramienta para calcular el área bajo la gráfica de una función integrable $f$. El cálculo en varias variables se extiende esta idea a funciones de $\mathbb{R}^n$ en $\mathbb{R}$ y se muestra que para calcular \enquote{el} volumen de un conjunto acotado $A$ basta integrar la la función constante $1$ sobre $A$. Si las medidas en efecto generalizan la noción de volumen, también generalizan a la integral. Introducimos primero una clase sencilla de funciones integrables, y una integral intuitiva pero restringida a estas funciones que extendemos paso a paso sin demostraciones.

\begin{definition}\label{def:int}
Sea $(\Omega, \mathcal{S}, \mu)$ un espacio de medida. Una función no negativa $\varphi:\Omega\to\mathbb{R}$ es \textit{simple} si sólo toma un número finito de valores. Es decir, $\varphi=\sum_{i=1}^na_i\chi_{_{A_i}}$ con los $A_i$ disjuntos. La \textit{integral de $\varphi$ con respecto a $\mu$} es
\begin{equation*}
	\int\varphi d\mu := \sum_{i=1}^n a_i\mu(A_i)
\end{equation*}
Extendemos la integral a funciones medibles no negativas $g$ tomando
\begin{equation*}
	\int g d\mu = \sup_{\varphi \leq g}\int fd\mu
\end{equation*}
 Finalmente, si usando la descomposición $f=f^+ -f^-$ con $f^+(\omega)=\max\left(f(\omega), 0\right)$ y $f^-(\omega)=\max\left(-f(\omega), 0\right)$ se satisface que $\int f^+d\mu, \int f^-d\mu < \infty$, se dice que $f$ es \textit{integrable}, y
 \begin{equation*}
 \int f d\mu = 	\int f^+d\mu - \int f^-d\mu
 \end{equation*}
\end{definition}

\begin{definition}
	La integral de una función $\mu$-integrable $f$ sobre un conjunto $A$ es
	\begin{equation*}
	\int_A fd\mu = \int f\cdot\chi_a d\mu	
	\end{equation*}
\end{definition}

\begin{theorem}\label{thm:props-int}
La integral de Lebesgue satisface las siguientes propiedades.
\begin{enumerate}[label=\roman*.]
	\item $\int (f+g)d\mu = \int f d\mu + \int g d\mu$
	\item $\int(\alpha f)d\mu = \alpha \int f d\mu$
	\item $f \leq g \; \mu\text{-c.d} \Rightarrow \int fd\mu \leq \int gd\mu$
	\item $f = g\; \mu\text{-c.d} \Rightarrow \int fd\mu = \int gd\mu$
	\item $\mu(A)=0\Rightarrow \int_Afd\mu=0$
\end{enumerate}	
\end{theorem}

\begin{prop}\label{thm:jensen}
Sea $\varphi$ una función convexa.
\begin{equation*}
			\varphi\left(\int fd\mu \right) \leq \int\left(\varphi\circ f\right)d\mu
\end{equation*}
\end{prop}


Calcular el volumen usual es integrar con respecto a la medida de Lebesgue, pero la definición que dimos es más general. Supongamos que conocemos una medida $\mu$ en $\Omega$, y sea $f$ una función no-negativa. Puede probarse que el mapeo
\begin{equation*}
	\nu(A)=\int_Afd\mu 
\end{equation*} 
es una medida en $\Omega$. Expresiones como la del lado derecho deberían resultar familiar de un curso de cálculo de probabilidades, pues cuando $\Omega=\mathbb{R}$ y $\mu=\lambda$, las funciones integrables no-negativas son exactamente densidades de probabilidad. Esto significa que la probabilidad $P(X\in [a,b])=F(b)-F(a)$ ($F$ es una antiderivada de $F$) es en realidad una \textit{medida de probabilidad} en $\mathbb{R}$ y la densidad $f$ es la manera de cambiar entre la medida de Lebesgue y la medida \enquote{probabilidad de $X$}, $\nu$. 

\begin{definition}
$\nu$ es \textit{absolutamente continua con respecto a} $\mu$, denotado $\nu \ll \mu$, si para todo conjunto medible $A$, $\mu(A)=0 \Rightarrow \nu(A)=0$. 
\end{definition}

La discusión de arriba sugiere que dadas una medida y una función no-negativa podemos generar una medida absolutamente continua con respecto la primera usando las propiedades del teorema \ref{thm:props-int}. El converso también es cierto.

\begin{theorem}[Radon-Nikodym]
Sea $(\Omega, \mathcal{S})$ un espacio medible y $\mu, \nu$ dos medidas $\sigma$-finitas en él. $\nu \ll \mu$ si y sólo si existe una función medible $f$ tal que para todo $A \in \mathcal{S}$
\begin{equation}\label{eq:rn-thm}
	\nu(A) = \int_A f d\mu
\end{equation}
Más aún, $f$ está definida de manera única casi dondequiera.
\end{theorem}

\begin{definition}
A la función $f$ de la ecuación \eqref{eq:rn-thm} se le llama \textit{derivada de Radon-Nikodym de $\nu$ con respecto a $\mu$}, y para asemejar la notación de cálculo diferencial usamos 
\begin{equation*}
f = \frac{d\nu}{d\mu}	
\end{equation*}
o por conveniencia , $d\nu = fd\mu$. Cuando $\nu$ es una medida de probabilidad, se le llama a $f$ \textit{función de densidad de probabilidad generalizada} o f.d.p.g.
\end{definition}

\begin{prop}\label{thm:props-rn}
Las derivadas de Radon-Nikodym satisfacen
\begin{enumerate}[label=\roman*.]
	\item aditividad
		\begin{equation*}
			\frac{d(\mu+\nu)}{d\lambda} = \frac{d\mu}{d\lambda}+\frac{d\nu}{d\lambda}
		\end{equation*}
	\item regla de la cadena
		\begin{equation*}
			\frac{d\mu}{d\lambda} = \frac{d\mu}{d\nu}\cdot\frac{d\nu}{d\lambda}
		\end{equation*}
\end{enumerate}
\end{prop}


Los cambios de medida se relacionan con la integral a través de la siguiente ecuación.

\begin{lemma}\label{lemma:trans-int}
	Sean $(\Omega, \mathcal{S}, \mu)$ un espacio de medida, $(\Xi, \mathcal{E})$ un espacio medible y $T:\Omega \to \Xi$ medible. Para toda función $g:\Xi \to \overline{\mathbb{R}}$ y todo $B\in\mathcal{E}$, se cumple que
	\begin{equation*}
		\int _Bg\ dT_*\mu=\int_{T\in B}(g\circ T)d\mu
	\end{equation*} 
	donde $T_*\mu$ es la medida \textit{pushforward} de $T$ con respecto a $\mu$,  
	\begin{equation*}
		T_*\mu(A):=\mu(T\in A)
	\end{equation*}
\end{lemma}
\begin{proof}
Basta probarlo para $B=\Xi$, pues para cualquier otro caso basta componer $g$ con la función característica $\chi_B$. Nótese que en la definición \ref{def:int} cualquier suma aproximadora del lado izquierdo es aproximadora del lado derecho y vice versa, pues por definición son de la forma
\begin{equation*}
	\sum_i g(\xi_i)T_*\mu(B_i) = \sum_i g\left(T(\omega_i)\right)\mu(T \in A)
\end{equation*}
por lo que tomando supremo, ambas integrales son siempre iguales.
\end{proof}

Las herramientas de teoría de la medida que acabamos de dar permiten definir de manera formal la probabilidad condicional.

\begin{definition}
Sean $(\Omega, \mathcal{S}, \mu)$ un espacio de medida, $f$ una función no negativa en $\Omega$  y $\nu$ la medida definida por $d\nu = fd\mu$. Si $T:\Omega \to (\Xi, \mathcal{E})$ es una función medible, la  \textit{esperanza condicional de $f$ dado que $T=\xi$} es la función $g$ tal que para todo $B\in\mathcal{E}$,

\begin{equation*}
	T_*\nu(B) = \int_B gdT_*\mu
\end{equation*}

Cuando $f=\chi_A$ para algún conjunto $A\in\mathcal{S}$, decimos $g$ llama \textit{probabilidad condicional de $A$}.
\end{definition}

En general se usa la notación $g(\xi) = f\left(\omega \mid T=\xi\right)$ porque si escribimos la ecuación completa tenemos que para todo $B\in\mathcal{E}$

\begin{equation*}
	\int_{T\in B} f(\omega)\mu(d\omega)
	= \int _B f(\omega \mid T=\xi)T_*\mu(d\xi)
\end{equation*}

y para el caso de $f=\chi_A$

\begin{equation*}
\mu\left(A\cap T\in B\right) = 
	\int _B p(A \mid T=\xi)T_*\mu(d\xi) 	
\end{equation*}

que es exactamente la ecuación de probabilidad condicional como se define en cálculo de probabilidades.
\end{document}