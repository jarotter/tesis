\documentclass[main.tex]{subfiles}
\begin{document}
\chapter{Stein Variational Gradient Descent}
Usando la teoría de RKHS desarrollada en el capítulo 3, en este capítulo
construimos un algoritmo no paramétrico para inferencia variacional automática.
La última pieza necesaria es un resultado en teoría de aproximación debido a
Charles Stein.
% ----------------------
% ... STEIN'S METHOD ...
% ----------------------
\section{Del método de Stein a la discrepancia de Stein}

% ----------------------
% ........ SVGD ........
% ----------------------
\section{Stein Vartational gradient descent}
En la sección \ref{sec:advi} introdujimos la idea de hacer inferencia
variacional a través de transformaciones suaves. El algoritmo \eqref{algo:advi}
\enquote{estandariza} el problema variacional completando el soporte
$\mathcal{X}\subseteq{R}^d$ de la densidad de interés a $\mathbb{R}^d$ completo
y después minimiza $\kl$ con familias de normales, una solución paramétrica.
Este proceso utilliza una librería de transformaciones y sus jacobianos que el
equipo de desarrollo de Stan mantiene\cite{advi}, pero en principio no se
necesita salir de $\mathcal{X}$. La idea detrás de SVGD es partir de una
distribución $Q_0$ en $\mathcal{X}$ y construir de manera iterativa y no
paramétrica biyección suave $T: \mathcal{X} \to \mathcal{X}$ que minimice
$\kl(T_*Q_0\mmid P)$. 

En este capítulo usaremos la notación
\begin{equation*}
    q_T = \frac{dT_*Q}{d\lambda}
\end{equation*}

\begin{lemma}
    Sea $X\sim Q$ un vector aleatorio en $\mathcal{X}$ y $T_\epsilon:
    \mathcal{X} \to \mathcal{X}$ una biyección indexada por $\epsilon$ y
    diferenciable con respecto a $x$ y $\epsilon$.
    \begin{equation*}
        \nabla_\epsilon \kl(T_*Q\mmid P)  = 
        \mathbb{E}_{x\sim Q}\left[
            s_p(T(x))'\nabla_\epsilon T(x) + 
            \mathrm{traza}\left(\nabla_xT(x)\right)^{-1}\cdot
            \nabla_\epsilon \nabla_x T(x)
        \right] 
    \end{equation*}
    donde $s_p = \nabla\log p$.
\end{lemma}
\begin{proof} $ $ \newline
    Usando el corolario \ref{cor:invar-invert}, 
    \begin{align*}
        \nabla_\epsilon\kl(T_*Q\mmid P) &= \kl(Q \mmid T^{-1}P) \\
        &:= \nabla_\epsilon\int\log\frac{dQ}{dT^{-1}_*P}dQ \\
        &= \int\nabla_\epsilon\log\frac{dQ}{dT^{-1}P}dQ
    \end{align*}

\end{proof}
\end{document}